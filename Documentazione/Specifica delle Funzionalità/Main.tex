\documentclass[italian, 12pt, a4paper]{article}
\usepackage{graphicx} %Per inserire immagini
\usepackage[table]{xcolor}
\usepackage[hidelinks]{hyperref} %Evita bordi colorati nei link

\setlength{\parindent}{10mm} %Imposta il rientro della prima riga di ogni paragrafo
\usepackage{xcolor}
\usepackage{titlesec}

\titleformat{\section}{\color{violet!60}\normalfont\Large\bfseries}{\thesection}{1em}{} %Imposta il colore delle section
\titleformat{\subsection}{\color{violet!45}\normalfont\large\bfseries}{\thesubsection}{1em}{} %Imposta il colore delle subsection
\titleformat{\subsubsection}{\normalfont\normalsize\bfseries\color{violet!30}}{\thesubsubsection}{1em}{} %Imposta il colore delle subsubsection
\title{\huge{HomeProject}}
\author{Matteo Saramin, Filippo Sperandio \textit{DomusFelix} \\ {\small ITTS Vito Volterra} \\ \\ \emph{Cliente: Roberto Rossi}}

\newcommand{\version}{Version 2.5}
\date{\version\\ 15 Marzo 2025}
\begin{document}

\maketitle % Mostra titolo, autore e data

\section{Storico del Documento}

\begin{center}
    \renewcommand{\arraystretch}{1.5} % Aumenta lo spazio verticale
    \rowcolors{2}{violet!30}{} % Alterna colori a partire dalla seconda riga
    \begin{tabular}{|c|c|c|c|}
        \hline
        \rowcolor{violet!30}
        Versione & Data & Autore & Nome Documento \\
        \hline
        1.0 & 14/03/2025 & Matteo & main.tex \\
        \hline
        2.0&25/03/2025&Matteo e Filippo & main.tex \\
        \hline
        2.5&28/03/2025&Matteo e Filippo&Main.tex\\
        \hline
    \end{tabular}\\[4mm]
\end{center}
\vspace{15mm}
\clearpage
\section{Sommario}
\begin{enumerate}
    \item \hyperref[sec:filosofia]{\Large Filosofia dell'Azienda}
    \item \hyperref[sec:introduzione]{\Large Introduzione}
    \item \hyperref[sec:situazione]{\Large Situazione attuale}
    \item \hyperref[sec:requisiti]{\Large Requisiti Hardware \& Software}
        \begin{enumerate}
            \item Dispositivi di Illuminazione
            \item Dispositivi per la Climatizzazione
            \item Dispositivi Impianto Fotovoltaico
        \end{enumerate}
    \item \hyperref[sec:requisiti2]{\Large Specifiche Requisiti Funzionali}
        \begin{enumerate}
            \item HMI
            \item Specifiche Funzionalità
            \item Progettazione di Rete
        \end{enumerate}
\end{enumerate}
\clearpage
\section{Filosofia dell'Azienda}\label{sec:filosofia}
Per garantire che il sistema realizzato sia il più possibile aderente alle necessità del cliente, abbiamo strutturato un questionario mirato, suddiviso in diverse sezioni. Questo questionario è stato elaborato per raccogliere dettagli sulla configurazione dell’abitazione, i dispositivi domotici desiderati, le funzionalità ritenute più importanti e le preferenze riguardanti l’interfaccia utente.
Le informazioni fornite dal cliente ci permetteranno di sviluppare un sistema domotico su misura, ottimizzando la disposizione dei dispositivi, i parametri di controllo e l’interazione con il sistema. Inoltre, l’analisi delle risposte ci aiuterà a progettare un’interfaccia grafica chiara e funzionale, organizzata in base alle diverse zone della casa e alle principali operazioni disponibili.\\[1.5mm]
La comunicazione con il cliente è un elemento essenziale per il successo del progetto, e questo documento rappresenta un primo passo cruciale per garantire che il prodotto finale risponda alle aspettative e necessità dell’utente.
\clearpage
\section{Introduzione}\label{sec:introduzione}
Il progetto seguente prevede lo sviluppo di un’intero sistema domotico destinato all'abitazione dei nostri clienti. Il sistema nella sua interezza prevede un'interfaccia grafica interattiva che consentirà agli utenti di controllare in modo semplice ed efficace le funzionalità della propria abitazione; una progettazione di rete simulando in modo reale i dispositivi, funzionalità e scenari, e infine i testi documentativi del progetto. L’idea centrale è offrire un sistema intuitivo, accessibile da più dispositivi e in grado di migliorare il comfort, la sicurezza e l’efficienza energetica dell'abitazione.\\[1.5mm]
\clearpage
\section{Situazione attuale}\label{sec:situazione}
\subsection{Dispositivi Presenti}
In questa sezione analizzeremo i dispositivi già presenti nell'edificio domestico dei clienti, prima del nostro intervento. Questi dispositivi, acquistati e installati precedentemente, comprendono solo 2 tipologie di corpi illuminanti, abbimao dovuto installarne un terzo sotto specifica richiesta del cliente, condizionatori, radiatori, irrigatori, finestre e termostati.
\subsection{Illuminazione}
\subsubsection{1° Lampada Pendente}
Abbiamo scelto come corpo illuminante principale un lampadario pendente a LED, di forma sferica e paralume in tessuto ignifugo, adatta ad un'illuminazione ottimale di vasti ambienti o camere.
\begin{quote}
    \begin{enumerate}
        \item Sorgente Luminosa: \emph{LED}.
        \item Potenza: \emph{50W}.
        \item Flusso luminoso: \emph{4000 lumen}.
        \item Temperatura Colore: \emph{4000K}.
        \item D = \emph{50cm} $H_t = \emph{190cm}$
    \end{enumerate}
\end{quote}
\begin{figure}[h!]
    \centering
    \begin{minipage}{0.45\textwidth}
        \centering
        \includegraphics[width=0.6\linewidth]{img/lampada.png}
        \caption{Vista 3D}
    \end{minipage} \hfill
    \begin{minipage}{0.45\textwidth}
        \centering
        \includegraphics[width=0.6\linewidth]{img/lampada-top.png}
        \caption{Vista dall'alto}
    \end{minipage}
\end{figure}
\hrule
\subsubsection{2° Lampada da Muro}
Abbiamo scelto come secondo corpo illuminante un lampadario pendente a LED, di forma sferica e paralume in tessuto ignifugo, adatta ad un'illuminazione ottimale di vasti ambienti o camere.
\begin{quote}
    \begin{enumerate}
        \item Sorgente Luminosa: \emph{LED}.
        \item Potenza: \emph{28W}.
        \item Flusso luminoso: \emph{4000 lumen}.
        \item Temperatura Colore: \emph{3000K}.
        \item L = \emph{60cm}; $H_t = \emph{150cm};$
    \end{enumerate}
    \vspace{-40pt}
\end{quote}
\begin{figure}[h!]
    \centering
    \begin{minipage}{0.45\textwidth}
        \centering
        \includegraphics[width=\linewidth]{img/Lampada_muro.png}
        \caption{Vista 3D}
    \end{minipage} \hfill
    \begin{minipage}{0.45\textwidth}
        \centering
        \includegraphics[width=\linewidth]{img/lampada-muro-top.png}
        \caption{Vista dall'alto}
    \end{minipage}
\end{figure}
\hrule
\clearpage
\subsection{Dispositivi per la Climatizzazione}
\subsubsection{Termostato}
Tramite il termostato il cliente ha la possibilità di regolare in maniera autonoma la temperatura della casa. Sono 4 le modalità possibli: Spento, Cool, Heat, Auto.
\begin{quote}
    \begin{enumerate}
        \item Modalità: \emph{Off, Cool, Heat, Auto}
    \end{enumerate}
\end{quote}
\begin{figure}[h!]
    \centering
    \begin{minipage}{0.45\textwidth}
        \centering
        \includegraphics[width=0.7\linewidth]{img/thermostat_icon_off.png}
        \caption{Vista 3D}
    \end{minipage} \hfill
    
\end{figure}

\subsubsection{1° Dispositivo Termoregolatore}
Ho scelto di installare un condizionatore con funzione sia di raffreddamento che di riscaldamento per garantire un comfort ottimale tutto l'anno, ottimizzando l'efficienza energetica e riducendo i costi rispetto a soluzioni separate.
\begin{quote}
    \begin{enumerate}
        \item Capacità di Raffredamento: \emph{9000 BTU}
        \item Capacità di Riscaldamento: \emph{7000 BTU}
        \item Efficienza Energetica: \emph{A++}
        \item W = \emph{48cm};  = \emph{85cm}; Peso: \emph{10kg}; 
    \end{enumerate}
\end{quote}
\begin{figure}[h!]
    \centering
    \begin{minipage}{0.45\textwidth}
        \centering
        \includegraphics[width=0.7\linewidth]{img/condizionatore.png}
        \caption{Vista 3D}
    \end{minipage} \hfill
    \begin{minipage}{0.45\textwidth}
        \centering
        \includegraphics[width=0.7\linewidth]{img/condizionatore-top.png}
        \caption{Vista dall'alto}
    \end{minipage}
\end{figure}
\hrule
\subsubsection{2° Dispositivo di Riscaldamento}
Abbiamo scelto di installare un pannello riscaldante in bagno per garantire un riscaldamento più uniforme e discreto. Inoltre, il pannello offre un'alternativa più pulita e sicura rispetto ai tradizionali radiatori, riducendo al minimo l'accumulo di polvere e aumentando l'efficienza energetica.
\begin{quote}
    \begin{enumerate}
        \item Capacità di Riscaldamento: \emph{7000 BTU}
        \item Efficienza Energetica: \emph{A++}
        \item W = \emph{102cm}; L = \emph{64cm}; Peso: \emph{7kg}; 
    \end{enumerate}
\end{quote}
\vspace{-15px}
\begin{figure}[h]
    \centering
    \begin{minipage}{0.45\textwidth}
        \centering
        \includegraphics[width=0.4\linewidth]{img/pannelo.png}
        \caption{Vista 3D}
    \end{minipage} \hfill
    \begin{minipage}{0.45\textwidth}
        \centering
        \includegraphics[width=0.4\linewidth]{img/panello-top.png}
        \caption{Vista dall'alto}
    \end{minipage}
\end{figure}
\clearpage

\clearpage
\section{Requisiti Hardware \& Software}\label{sec:requisiti}
Per un coretto ed efficiente funzionamento dell'intero sistema domotico della casa del nostro cliente, in relazione alle sue richieste e necessità, abbiamo individuato i seguenti requisiti hardware e software.
\begin{enumerate}
    \item L'interfaccia grafica deve essere utilizzata dal cliente, e richiede che sia chiara, semplice e intuitiva, in modo da modificare i parametri della casa a suo piacimento. Questi parametri sono: la temperatura, l'illuminazione, l'automazione delle finestre, l'irrigazione del giardino e il sistema d'allarme.
    \item L’interfaccia deve adattarsi a diversi dispositivi, funzionando efficacemente su desktop e mobile.
    \item Le modifiche apportate dall’utente devono essere applicate in tempo reale e riflesse nel sistema senza ritardi eccessivi.
    \item Il sistema deve supportare i dispositivi intelligenti che implementano le funzionalità scelte dal'utente.
    \item Il sistema deve permettere di accendere e spegnere le luci in ogni stanza, sia tramite l'interfaccia grafica che fisicamente tramite un pulsante; inoltre sono implementanti delle automazioni, in relazionea a sensori e orari.
    \item Il sistema deve permettere di gestire la temperatura della casa, tramite gli adeguati dispositivi di riscaldamento e raffreddamento.
    \item Il sistema deve supportare l'impianto fotovoltaico, munito di 7 pannelli solari e una batteria, per una maggiore efficienza e garantendo l’utilizzo di energia completamente rinnovabile.
    \item Deve essere possibile verificare la produzione energetica dei pannelli fotovoltaici. Di conseguenza è necessario visualizzare anche la percentuale della batteria. Gli altri parametri necessari sono il consumo energetico dell’intera casa e la temperatura interna e esterna della casa.
\end{enumerate}
\clearpage
Inoltre per implementare le funzionalità richieste è stato neccesario acquistare e installare diversi dispositivi. 
\subsection{Dispositivi Installati}
\subsubsection{Corpi illuminanti}
\paragraph{3° Lampada da Tavolo}
Come terzo corpo illuminante abbiamo scelto una lampada da tavolo, combaciando perfettamente un ottimo oggetto di design e una illuminazione soffusa nelle ore notturne perfetta per la lettura, scrittura o altre attività.
\begin{quote}
    \begin{enumerate}
        \item Sorgente Luminosa: \emph{LED}.
        \item Potenza: \emph{100W}.
        \item Flusso luminoso: \emph{1600 lumen}.
        \item Temperatura Colore: \emph{2700K}.
        \item D = \emph{42cm}; H = \emph{62cm};
        \item Costo: 43,00€
        \item N°: 1.
    \end{enumerate}
\end{quote}
\begin{figure}[h]
    \centering
    \begin{minipage}{0.45\textwidth}
        \centering
        \includegraphics[width=0.5\linewidth]{img/lampada-tavolo.png} % modifica con il percorso della tua prima immagine
        \caption{Vista 3D}
    \end{minipage} \hfill
    \begin{minipage}{0.45\textwidth}
        \centering
        \includegraphics[width=0.5\linewidth]{img/lampada-tavolo-top.png} % modifica con il percorso della tua seconda immagine
        \caption{Vista dall'alto}
    \end{minipage}
\end{figure}
\clearpage
\subsubsection{Dispositivi Impianto Fotovoltaico}
\paragraph{Pannello Fotovoltaico}
Ho scelto il pannello JA Solar Black Frame da 415W per la sua alta efficienza e affidabilità. La potenza di 415W garantisce una buona produzione di energia, mentre la garanzia di 12 anni assicura una lunga durata. Il pannello utilizza tecnologia monocristallina, offrendo un'alta efficienza nella produzione di energia solare anche nelle ore con scarsa luminosità.
\begin{quote}
    \begin{enumerate}
        \item Potenza nominale: \emph{410W}.
        \item N° di Celle: \emph{108 ($6\times 18$)}.
        \item T° Operativa: \emph{-40° - +85°}.
        \item Garanzia: \emph{12 anni}.
        \item L = \emph{172cm}; W = \emph{97cm}; peso: \emph{16.7kg};
        \item Costo: 94€ per pannello.
        \item N°: 7.
    \end{enumerate}
\end{quote}

\begin{figure}[h]
    \centering
    \includegraphics[width=0.5\linewidth]{img/pannello-solare.png}
    \caption{Vista 3D}\label{fig:pannello-solare}
\end{figure}
\clearpage
\paragraph{Batteria}
La scelta della batteria per l'impianto fotovoltaico è stata dettata da parametri rigurdanti una totale sicurezza, data la sensibilità del dispositivo. La SolarEdge Home Battery assicura un'alimentazione della casa anche nelle ore con scarsa o nulla produzione energetica, grazie alla qualità del prodotto. Abbiamo deciso di installare due batterie, consci dell'aumento del prezzo, per un'autonomia energetica totale della casa. 
\begin{quote}
    \begin{enumerate}
        \item Energia Totale: \emph{10300Wh}.
        \item potenza uscita Contnua: \emph{5000W}.
        \item Grado di Protezione: \emph{IP55}.
        \item T° Operativa: \emph{-10° - +50°}.
        \item Garanzia: \emph{10 anni}.
        \item L = \emph{1179cm}; W = \emph{250cm}; peso: \emph{121kg};
        \item Costo: 4.455,00€
        \item N°: 2.
    \end{enumerate}
\end{quote}
\begin{figure}[h]
    \centering
    \includegraphics[width=0.6\linewidth]{img/batteria.png}
    \caption{Vista 3D}\label{fig:batteria}
\end{figure}
\clearpage
\subsubsection{Sensori}
\paragraph{Sensore di movimento}
Per garantire una protezione completa sia interna che esterna, abbiamo scelto il Dahua ARD2251E-W2(868). Questo sensore combina la tecnologia PIR e microonda, offrendo una copertura affidabile per ambienti interni ed esterni. La sua protezione IP65 lo rende resistente agli agenti atmosferici, mentre la doppia tecnologia assicura un rilevamento preciso e una riduzione dei falsi allarmi. È la soluzione ideale per un sistema di sicurezza integrato e versatile, con la possibilità di monitorare senza problemi sia ambienti interni che esterni.
\begin{quote}
    \begin{enumerate}
        \item Rilevamento: \emph{Tecnologia PIR \& microonda}.
        \item Portata: \emph{12 m, angolo di 90°}
        \item Alimentazione: \emph{Batteria CR123A}
        \item Protezione \emph{IP65}
        \item Costo: 38,00€
        \item N°: 4.
    \end{enumerate}
\end{quote}
\begin{figure}[h]
    \centering
    \begin{minipage}{0.45\textwidth}
        \centering
        \includegraphics[width=1\linewidth]{img/sensore_mov.png}
        \caption{Vista 3D}
    \end{minipage} \hfill
    \begin{minipage}{0.45\textwidth}
        \centering
        \includegraphics[width=0.4\linewidth]{img/sensoreC.png}
        \caption{Vista dall'alto}
    \end{minipage}
\end{figure}
\clearpage
\paragraph{Rilevatore del Vento}
Abbiamo scelto il sensore di vento BFT ANEM P111182 per la sua affidabilità nel monitorare la velocità del vento. Questo anemometro consente di automatizzare la chiusura delle finestre in caso di venti forti, evitando danni e migliorando la sicurezza.
\begin{quote}
    \begin{enumerate}
        \item Tipologia: \emph{Anemometro ad encoder, 2 impulsi per giro}.
        \item Alimentazione: \emph{24 V}
        \item Costo: 54,00€
        \item N°: 2.
    \end{enumerate}
\end{quote}
\begin{figure}[h]
    \centering
    \begin{minipage}{0.45\textwidth}
        \centering
        \includegraphics[width=1\linewidth]{img/sensore_vento.png}
        \caption{Vista 3D}
    \end{minipage} \hfill
    \begin{minipage}{0.45\textwidth}
        \centering
        \includegraphics[width=0.4\linewidth]{img/sensore_ventoC.png}
        \caption{Vista dall'alto}
    \end{minipage}
\end{figure}
\clearpage
\subsubsection{Sistema d'allarme}
\paragraph{Sirena}
Abbiamo optato per la sirena d'allarme da esterno Dahua ARA13-W2\-868 per la sua affidabilità e resistenza. La tecnologia wireless a lunga portata e la comunicazione bidirezionale sicura garantiscono un'installazione semplice e flessibile, ideale per sistemi domotici. La protezione IP65 e la doppia alimentazione la rendono ideale per un utilizzo esterno in qualsiasi condizione.
\begin{quote}
    \begin{enumerate}
        \item Pressione Sonora: \emph{110dB}
        \item Portata: \emph{fino a 1600 m in spazio aperto.}
        \item Alimentazione: \emph{A batteria CR123A}
        \item Protezione: \emph{IP65}
        \item Costo: 96,15€
        \item N°: 2
    \end{enumerate}
\end{quote}
\begin{figure}[h]
    \centering
    \begin{minipage}{0.45\textwidth}
        \centering
        \includegraphics[width=0.4\linewidth]{img/sirena.jpg}
        \caption{Vista 3D}
    \end{minipage} \hfill
    \begin{minipage}{0.45\textwidth}
        \centering
        \includegraphics[width=0.4\linewidth]{img/sirenaC.png}
        \caption{Vista dall'alto}
    \end{minipage}
\end{figure}
\clearpage
\paragraph{Telecamera}
Per completare il sistema di sicurezza abbiamo scelto la telecamera PIR wireless Dahua ARD1731-W2(868), della stessa serie e marca della sirena d'allarme. Questo garantisce massima compatibilità, comunicazione diretta e una gestione integrata tramite lo stesso ecosistema Dahua AirShield.
\begin{quote}
    \begin{enumerate}
        \item Risoluzione: \emph{1600 $\times$ 1200 pixel}
        \item Autonomia: \emph{batteria CR123A}
        \item Sicurezza: \emph{crittografia AES128}
        \item Costo: 93,00€
        \item N°: 2.
    \end{enumerate}
\end{quote}
\begin{figure}[h]
    \centering
    \begin{minipage}{0.45\textwidth}
        \centering
        \includegraphics[width=1\linewidth]{img/telecamera.png}
        \caption{Vista 3D}
    \end{minipage} \hfill
    \begin{minipage}{0.45\textwidth}
        \centering
        \includegraphics[width=0.3\linewidth]{img/telecameraC.png}
        \caption{Vista dall'alto}
    \end{minipage}
\end{figure}

    % Specifica Requisiti Funzionali
\clearpage
\section{Specifica Requisiti Funzionali}\label{sec:requisiti2}
\subsection{HMI - Human Machine Interface}
Il cliente richiede un’interfaccia grafica (HMI) accessibile da più dispositivi, tra cui computer, telefoni, un tablet e pannelli di controllo installati in ogni stanza. Questi dispositivi devono garantire un'esperienza di uso ottimale, un'interfaccia responsive che si adatti automaticamente alle diverse dimensioni dello schermo e che mantenga la navigazione fluida e intuitiva.  I pannelli di controllo posizionati strategicamente in ogni stanza della casa, devono consentire la gestione dei parametri della relativa stanza, di cui: illuminazione, temperatura, stato delle finestre, irrigazione e il sistema d’allarme. Gli altri dispositivi, di cui telefoni, computer e un tablet devono invece rappresentare lo stato generale della casa, con comunque la possibilità di interagire con il sitema domotica di ogni singola stanza.\\[1.5mm]
Un aspetto fondamentale dell’HMI è il monitoraggio dell’impianto fotovoltaico, che deve fornire in tempo reale dati sulla produzione energetica, la percentuale di carica della batteria e il consumo energetico dell’abitazione. Il cliente desidera una rappresentazione chiara e dettagliata di queste informazioni, con grafici intuitivi e indicatori che permettano di valutare rapidamente lo stato e l’efficienza del sistema.\\[1.5mm]
La struttura dell’interfaccia grafica richiesta dall’utente è la seguente: un prima rappresentazione visiva dei parametri fondamentali della casa domotica, visti in precedenza, successivamente una rappresentazione dell’intera casa, munita degli apparati intelligenti, in modo da modificare lo stato di essi. Infine un’ultima sezione con alcune specifiche del complesso della casa; di cui queste sono richieste esplicitamente dall’utente: Consumo energetico dell’intera casa, produzione in kW dell’impianto fotovoltaico e lo stato di carica della batteria, e la temperature esterna ed interna.\\[1.5mm]
La modalità con cui l’utente approccia con il sistema è appunto attraverso l’interfaccia grafica, deve quindi quest’ultima essere semplice, intuitiva, senza eccessivi ritardi e un’organizzazione delle funzioni che renda il controllo della casa semplice e immediato, anche per utenti senza particolari competenze tecniche.
\subsection{Specifiche Funzionalità}
\ldots
\clearpage
\end{document}
