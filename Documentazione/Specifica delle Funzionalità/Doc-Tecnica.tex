\documentclass[italian, 12pt, a4paper]{article}
\usepackage{graphicx} % Required for inserting images
\usepackage[table]{xcolor}
\usepackage[hidelinks]{hyperref} % Evita bordi colorati nei link

\setlength{\parindent}{10mm} % Imposta il rientro della prima riga di ogni paragrafo
\usepackage{xcolor}
\usepackage{titlesec}

\titleformat{\section}{\color{violet!60}\normalfont\Large\bfseries}{\thesection}{1em}{}
\titleformat{\subsection}{\color{violet!45}\normalfont\large\bfseries}{\thesubsection}{1em}{}
\titleformat{\subsubsection}{\normalfont\normalsize\bfseries\color{violet!30}}{\thesubsubsection}{1em}{}
\title{\huge{HomeProject}}
\author{Matteo Saramin, Filippo Sperandio \textit{DomusFelix} \\ {\small ITTS Vito Volterra} \\ \\ \emph{Cliente: Roberto Rossi}}

\newcommand{\version}{Version 1.0} % Definisce il comando per la versione
\date{\version\\ 15 Marzo 2025}
\begin{document}
\maketitle
\section{Storico del Documento}
\begin{center}
    \renewcommand{\arraystretch}{1.5} % Aumenta lo spazio verticale
    \rowcolors{2}{violet!30}{} % Alterna colori a partire dalla seconda riga
    \begin{tabular}{|c|c|c|c|}
        \hline
        \rowcolor{violet!30}
        Versione & Data & Autore & Nome Documento \\
        \hline
        1.0 & 14/04/2025 & Matteo e Filippo & Doc-Tecnica.tex \\
        \hline
    \end{tabular}\\[4mm]
\end{center}
\clearpage
\section{Sommario}
\begin{enumerate}
    \item \hyperref[sec:progettazione]{\Large Progettazione di Rete}
    \item \hyperref[sec:dispositivi]{\Large Dispositivi}
    \begin{enumerate}
        \item \hyperref[sec:lan]{Dispositivi Rete LAN}
        \item \hyperref[sec:domotico]{Dispositivi Sistema Domotico}
    \end{enumerate}
\end{enumerate}
\clearpage
\section{Progettazione di Rete}\label{sec:progettazione}
Abbiamo progettato, in base alle caratteristiche dell'abitazione fornitaci, un sistema  che permetta, come concordato con l’utente, di: 

-Regolare la temperatura utilizzando termosifoni presenti in ogni stanza (esclusi cabina armadio, cucina, studio e corridoi) per il riscaldamento, e per il raffreddamento mediante l’utilizzo di 3 condizionatori (presenti in salotto, studio e corridoio primo piano);
Per l'impostazione della temperatura l’utente ha a disposizione due pannelli fisici presenti in salotto al piano terra e in corridoio nel primo piano e la sezione apposita nell’interfaccia grafica.

-Aprire o chiudere le finestre mediante l’interfaccia grafica nella sezione apposita.
La chiusura delle finestre è automatizzata in base alla rilevazione del dispositivo “wind detector”, questo decide di chiudere le finestre in caso di vento troppo forte;
vengono chiuse anche nel caso in cui il termostato venga azionato per permettere un cambio di temperatura più rapido e un minore consumo energetico.

-L’accensione delle luci è controllata mediante l’interfaccia grafica, queste sono posizionate in base in questo modo:
giardino: 2;
piano terra: 8 (cucina: 3, entrata: 2, cucina: 1, studio: 1, bagno: 1);
primo piano: 7 (una per stanza + 1 in balcone)

-L'installamento di un sistema di allarme costituito da:
Sirena: piazzata in entrata esternamente, 
Telecamera: 1 piazzata in entrata esternamente;
Sensori di movimento: 4 piazzati in punti strategici quali: entrata esternamente, Corridoio piano terra, corridoio primo piano e balcone..
Il sistema è attivabile mediante la sezione apposita nell’interfaccia grafica.

-L'installazione di un sistema di irrigazione per il giardino composto da 3 irrigatori azionabili mediante un timer nella sezione apposita creata nell’interfaccia grafica.

Inoltre Abbiamo implementato un sistema di rete completo composto da router, switch e vari dispositivi di rete per garantire la connettività e il funzionamento efficiente di tutto l'impianto domotico
-Home Gateway, switch e router sono piazzati al piano terra e questi offrono connessione ad 1 pc fisso e ad 1 laptop;
-al piano superiore è presente un Access Point che fornisce connessione a 3 telefoni.
\clearpage
\section{Dispositivi della Rete}\label{sec:dispositivi}
\subsection{Dispositivi utilizzati per la  LAN:}\label{sec:lan}
L’indirizzamento della rete LAN è stato strutturato utilizzando il range 192.168.1.0/24, una rete privata comunemente usata nelle reti locali. Al router è stato assegnato l’indirizzo 192.168.1.1, mentre agli altri dispositivi (PC, switch, access point, smartphone) sono stati assegnati indirizzi statici all’interno dello stesso intervallo. Questa configurazione consente una comunicazione diretta tra tutti i dispositivi della rete LAN e un controllo più semplice degli indirizzi IP assegnati.
\begin{center}
    \renewcommand{\arraystretch}{1.5} % Aumenta lo spazio verticale
    \rowcolors{2}{violet!30}{} % Alterna colori a partire dalla seconda riga
    \begin{tabular}{|c|c|c|}
        \hline
        \rowcolor{violet!30}
        Dispositivo & Modello & Indirizzo IP\\
        \hline
        Home Gateway & DLC100 & 192.168.25.1\\
        \hline
        Router & ISR4331 & 192.168.1.1\\
        \hline
        Switch & 2900 - 24TT & 192.168.1.2\\
        \hline
        PC-Desktop & // & 192.168.1.3\\
        \hline
        PC-Laptop & // & 192.168.1.4\\
        \hline
        Access Point & AP-PT & 192.168.1.5\\
        \hline
        SmartPhone & Telefono1 & 192.168.1.6\\
        \hline 
        Smarthone & Telefono2 & 192.168.1.7\\
        \hline 
        Smarthone & Telefono3 & 192.168.1.8\\
        \hline
    \end{tabular}\\[4mm]
\end{center}
\clearpage
\subsection{Dispositivi utilizzati per il sistema domotico:}\label{sec:domotico}
\begin{center}
    \renewcommand{\arraystretch}{1.5} % Aumenta lo spazio verticale
    \rowcolors{2}{violet!30}{} % Alterna colori a partire dalla seconda riga
    \begin{tabular}{|c|c|c|}
        \hline
        \rowcolor{violet!30}
        Dipositivo IoT & MTBF* & Numero\\
        \hline
        Finestre & 87600 & 10\\
        \hline
        Luci & 43800 & 17\\
        \hline
        Radiatori & 262800 & 6\\
        \hline
        AC & 300000 & 3\\
        \hline
        Termostati & 43800 & 2\\
        \hline
        Irrigatori & 43800 & 3\\
        \hline
        Sirena & 26280 & 1\\
        \hline
        Sensore di Movimento & 26280 & 4\\
        \hline
        Telecamera & 26280 & 1\\
        \hline
    \end{tabular}\\[4mm]
\end{center}
\vspace{1cm}
Abbiamo deciso per questi dispositivi IoT, alcuni di essi implementati da noi appositamente per implemetare le funzionalità richieste, tra cui \emph{Luci, Sensori di Movimento e la Telecamera,} di non assegnare un indirizzo Ip Statico, e di configurare l'indirizzamento in DHCP automaticamente. In questo modo garantiamo ai clienti una maggiore sicurezza e semplicità nel complesso della rete.\\
Qual'ora i dispositivi IoT della rete non dovessero funzionare corretamente all'interno della simulazione, è necessario ricoleggarli al \emph{Home Server}, in modo da ripristinarli per un corretto funzionamento. I passaggi da seguire sono i seguenti: \\ \emph{Dispositivo IoT - Config - Home Server} \\
Il valore MTBF indica il grado di affidabilità di un dispositivo elettronico, meccanico o elettrico. Viene calcolato come la media del tempo che intercorre tra due guasti consecutivi durante il normale funzionamento.
\clearpage
\end{document}