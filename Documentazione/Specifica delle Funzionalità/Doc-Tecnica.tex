\documentclass[italian, 12pt, a4paper]{article}
\usepackage{graphicx} % Required for inserting images
\usepackage[table]{xcolor}
\usepackage[hidelinks]{hyperref} % Evita bordi colorati nei link

\setlength{\parindent}{10mm} % Imposta il rientro della prima riga di ogni paragrafo
\usepackage{xcolor}
\usepackage{titlesec}

\titleformat{\section}{\color{violet!60}\normalfont\Large\bfseries}{\thesection}{1em}{}
\titleformat{\subsection}{\color{violet!45}\normalfont\large\bfseries}{\thesubsection}{1em}{}
\titleformat{\subsubsection}{\normalfont\normalsize\bfseries\color{violet!30}}{\thesubsubsection}{1em}{}
\title{\huge{HomeProject}}
\author{Matteo Saramin, Filippo Sperandio \textit{DomusFelix} \\ {\small ITTS Vito Volterra} \\ \\ \emph{Cliente: Roberto Rossi}}

\newcommand{\version}{Version 2.5} % Definisce il comando per la versione
\date{\version\\ 15 Marzo 2025}
\begin{document}
\maketitle
\section{Storico del Documento}
La seguente tabella riporta lo storico delle versioni del documento, includendo la versione, la data di rilascio, gli autori delle modifiche e il nome del file corrispondente.

\begin{center}
    \renewcommand{\arraystretch}{1.5} % Aumenta lo spazio verticale
    \rowcolors{2}{violet!30}{} % Alterna colori a partire dalla seconda riga
    \begin{tabular}{|c|c|c|c|}
        \hline
        \rowcolor{violet!30}
        Versione & Data & Autore & Nome Documento \\
        \hline
        1.0 & 14/04/2025 & Matteo e Filippo & Doc-Tecnica.tex \\
        \hline
    \end{tabular}\\[4mm]
\end{center}
\clearpage
\section{Sommario}
\begin{enumerate}
    \item \hyperref[sec:introduzione]{\Large Introduzione}
    \item \hyperref[sec:progettazione]{\Large Progettazione di Rete}
\end{enumerate}
\clearpage
\section{Introduzione}\label{sec:introduzione}
Questo documento ha lo scopo di 
\clearpage
\section{Progettazione di Rete}\label{sec:progettazione}
owirngoegneneonge
\clearpage
\end{document}
