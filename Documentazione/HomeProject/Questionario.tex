\documentclass{exam}
\usepackage{graphicx} % Required for inserting images
\usepackage[table]{xcolor}
\usepackage{xcolor}
\usepackage{titlesec}

\titleformat{\section}{\color{violet!60}\normalfont\Large\bfseries}{\thesection}{1em}{}
\titleformat{\subsection}{\color{violet!45}\normalfont\large\bfseries}{\thesubsection}{1em}{}

\title{\huge{Questionario}}
\author{Matteo Saramin, Filippo Sperandio \textit{DomusFelix} \\ {\small ITTS Vito Volterra} \\ \\ \emph{Cliente: Roberto Rossi}}

\newcommand{\version}{Version 1.0}                   %Definisci il comando per la versione
\date{\version\\ 15 Marzo 2025}


\begin{document}
\maketitle
\vspace{5mm}
In questo documento presentiamo le domande rivolte al cliente, di carattere \emph{gestionale},
per comprendere al meglio le sue esigenze, necessità e preferenze. 
Questo ci permetterà di realizzare un prodotto il più possibile in linea con le sue aspettative. 
Riteniamo fondamentale questa comunicazione tra noi e il cliente e, grazie all'intervista svolta, abbiamo potuto redigere questo questionario.


\clearpage
\noindent\textbf{1. Come è strutturata la pianimetria della casa?}\\[1.5mm]
É importante per noi avere la pianimetria dell'edificio in modo da creare una rete efficiente e sicura.\\

\clearpage
\noindent\textbf{2. Quanti e quali tipi di dispositivi smart vorreste installare?}\\[1.5mm]
Inoltre è bene specificare in quali zone della casa andrebbero posizionati, e quali di questi devono esserre controllati dal sistema?\\

\clearpage
\noindent\textbf{3. Quali sono le principali funzioni che la casa domotica deve supportare?}\\[1.5mm]
Ad esempio l'illuminazione, sistemi d'allarme, climatizzazione\ldots\\[1.5mm]
\begin{quote}
\emph{Vorrei che la casa domotica supportasse diverse funzionalità per migliorare il comfort e la sicurezza. Mi piacerebbe avere un sistema di illuminazione intelligente. Inoltre, sarebbe utile un’apertura automatizzata delle finestre, che si adatti alle condizioni meteo, ad esempio chiudendole in caso di vento forte.
Un altro aspetto importante è l’irrigazione automatica del giardino, evitando sprechi d’acqua. Per il comfort interno, vorrei un sistema di regolazione della temperatura tramite l'interazione con l'utente. Infine, la sicurezza è essenziale: vorrei un sistema d’allarme con sensori di movimento posizionati nei punti nevralgici della casa, e una sirena di allarme per eventuali intrusioni.}\\
\end{quote}

\clearpage
\noindent\textbf{4. Quali parametri vorreste poter controllare?}\\[1.5mm]
Per esempio la temperatura, l'ulliminazione, finetre, controllo motorizzato delle finestre\ldots

\clearpage
\noindent\textbf{5. Con quali dispositivi vorresti interagire con il sistema?}\\[1mm]
Ad esempio Computer desktop, oppure dispositivi mobili come telefoni o tablet?\\[1.5mm]
\begin{quote}
\emph{Vorrei poter interagire con il sistema tramite tre telefoni, uno per ciascun componente della famiglia, così ognuno può gestire le funzioni domotiche in modo indipendente. Inoltre, mi piacerebbe avere un computer nello studio, utile per un controllo più dettagliato. Infine, un tablet sarebbe comodo da tenere in una zona comune della casa, come il soggiorno, per un accesso rapido alle funzioni principali.}\\
\end{quote}
\clearpage
\noindent\textbf{6. Quali caratteristiche ritiene importanti per l’interfaccia grafica del sistema?}\\[1.5mm]
\begin{quote}
\emph{Per l’interfaccia grafica, vorrei che fosse chiara e intuitiva, in modo da poter controllare tutto facilmente senza dover perdere tempo a cercare le funzioni. Mi piacerebbe un design moderno, con un’organizzazione ben suddivisa: ad esempio, avere le varie funzionalità separate per categoria (illuminazione, sicurezza, temperatura, ecc.) e una suddivisione per zone della casa, così da poter gestire ogni stanza in modo rapido.}
\end{quote}
\end{document}