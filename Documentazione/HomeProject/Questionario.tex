\documentclass{exam}
\usepackage{graphicx} % Required for inserting images
\usepackage[table]{xcolor}

\title{Questionario}
\author{Matteo Saramin, Filippo Sperandio \textit{DomusFelix} \\ {\small ITTS Vito Volterra} \\ \\ \emph{Cliente: Roberto Rossi}}

\newcommand{\version}{Version 1.0}                   %Definisci il comando per la versione
\date{\version\\ 15 Marzo 2025}


\begin{document}
\maketitle
\vspace{5mm}
In questo documento presentiamo le domande rivolte al cliente, di carattere \emph{gestionale},
per comprendere al meglio le sue esigenze, necessità e preferenze. 
Questo ci permetterà di realizzare un prodotto il più possibile in linea con le sue aspettative. 
Riteniamo fondamentale questa comunicazione tra noi e il cliente e, grazie all'intervista svolta, abbiamo potuto redigere questo questionario.

\clearpage
\noindent\textbf{1. Come è strutturata la pianimetria della casa?}\\[1.5mm]
É importante per noi avere la pianimetria dell'edificio in modo da creare una rete efficiente e sicura.\\

\vspace{15mm}
\noindent\textbf{2. Quanti e quali tipi di dispositivi smart vorreste installare?}\\[1.5mm]
Inoltre è bene speccificare in quali zone della casa andrebbero posizionati, e quali di questi devono esserre controllati dal sistema?\\

\vspace{15mm}
\noindent\textbf{3. Quali sono le principali funzioni che la casa domotica deve supportare?}\\[1.5mm]
Ad esempio l'illuminazione, sistemi d'allarme, climatizzazione\ldots

\vspace{15mm}
\noindent\textbf{4. Quali parametri vorreste poter controllare?}\\[1.5mm]
Per esempio la temperatura, l'ulliminazione, finetre\ldots

\vspace{15mm}
\noindent\textbf{5. Con quali dispositivi vorresti interagire con il sistema?}\\[1.5mm]
Ad esempio Computer desktop, oppure dispositivi mobili come telefoni o tablet?

\vspace{15mm}
\noindent\textbf{6. Quali caratteristiche ritiene importanti per l’interfaccia grafica del sistema?}\\[1.5mm]

\end{document}