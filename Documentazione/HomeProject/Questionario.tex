\documentclass{exam}
\usepackage{graphicx} % Required for inserting images
\usepackage[table]{xcolor}
\usepackage{xcolor}
\usepackage{titlesec}
\usepackage{hyperref}

\titleformat{\section}{\color{violet!60}\normalfont\Large\bfseries}{\thesection}{1em}{}
\titleformat{\subsection}{\color{violet!45}\normalfont\large\bfseries}{\thesubsection}{1em}{}

\title{\huge{Questionario}}
\author{Matteo Saramin, Filippo Sperandio \textit{DomusFelix} \\ {\small ITTS Vito Volterra} \\ \\ \emph{Cliente: Roberto Rossi}}

\newcommand{\version}{Version 1.0}                   %Definisci il comando per la versione
\date{\version\\ 15 Marzo 2025}


\begin{document}
\maketitle
\vspace{5mm}
In questo documento presentiamo le domande rivolte al cliente, di carattere \emph{gestionale},
per comprendere al meglio le sue esigenze, necessità e preferenze. 
Questo ci permetterà di realizzare un prodotto il più possibile in linea con le sue aspettative. 
Riteniamo fondamentale questa comunicazione tra noi e il cliente e, grazie all'intervista svolta, abbiamo potuto redigere questo questionario.


\clearpage
\noindent\textbf{1. Come è strutturata la pianimetria della casa?}\\[1.5mm]
É importante per noi avere la planimetria dell'edificio in modo da creare una rete efficiente e sicura.\\
\begin{quote}
    \emph{Abbiamo progettato la casa prediligendo comfort, estetica e funzionalità, con spazi aperti e ben distribuiti per favorire una buona circolazione dell'aria e della luce naturale. La disposizione delle stanze è stata pensata per ottimizzare l'uso degli spazi e garantire una rete efficiente e sicura in tutta la casa. La casa è struttura in due piani, il piano terra presenta ovviamente cucina, soggiorno, sala da pranzo, un bagno, un ripostiglio e uno studio: Il secondo piano comprende tre camere da letto, di cui una matrimoniale con cabina armadio e balcone, una per nostro figlio e una per gli ospiti; infine un bagno. Ovviamente tutte queste stanze sono dotate degli opportuni dispositivi intelligenti e dei rispettivi impianti. Per visualizzare al meglio la planimetria è stata progettata questo modello della casa, compreso di tutto l'arredamento reale: \href{https://gallery.roomsketcher.com/project/?l=en&pid=14670259}{\textcolor{blue}{Link}}
    }
\end{quote}
\clearpage
\noindent\textbf{2. Quanti e quali tipi di dispositivi smart vorreste installare?}\\[1.5mm]
Inoltre è bene specificare in quali zone della casa andrebbero posizionati, e quali di questi devono esserre controllati dal sistema?\\
\begin{quote}
    \emph{Sicuramente tutti i dispositivi necessari ad implementare le funzionalità richieste. Di conseguenza per l'illuminazione pensavamo ad dei lampadari pendenti a forma sferica, un'altra luce più discreta da inserire all'entrata o nel corridoio al piano superiore e infine una lampada in soggiorno. Abbiamo deciso di scartare l'opzione dei termosifoni, principalmente per una questione estetica, e abbiamo invece optato per dei condizionatori a muro, capaci sia di riscaldare che di raffredare. Mentre per i bagni un pannello riscaldante è più che adeguato, per una maggiore igiene ed estetica. Per il sistema d'allarme richiediamo dei sensori e una sirena per una totale sicurezza. Infine l'impianto di irrigazione deve comprendere dei semplici irrigatori, possibilmente a scomparsa.}
\end{quote}
\clearpage
\noindent\textbf{3. Quali sono le principali funzioni che la casa domotica deve supportare?}\\[1.5mm]
Ad esempio l'illuminazione, sistemi d'allarme, climatizzazione\ldots\\[1.5mm]
\begin{quote}
\emph{Vorrei che la casa domotica supportasse diverse funzionalità per migliorare il comfort e la sicurezza. Mi piacerebbe avere un sistema di illuminazione intelligente. Inoltre, sarebbe utile un’apertura automatizzata delle finestre, che si adatti alle condizioni meteo, ad esempio chiudendole in caso di vento forte.
Un altro aspetto importante è l’irrigazione automatica del giardino, evitando sprechi d’acqua. Per il comfort interno, vorrei un sistema di regolazione della temperatura tramite l'interazione con l'utente. Infine, la sicurezza è essenziale: vorrei un sistema d’allarme con sensori di movimento posizionati nei punti nevralgici della casa, e una sirena di allarme per eventuali intrusioni.}\\
\end{quote}

\clearpage
\noindent\textbf{4. Quali parametri vorreste poter visualizzare?}\\[1.5mm]
Per esempio la temperatura, l'ulliminazione, finetre, controllo motorizzato delle finestre\ldots
\begin{quote}
    \emph{Crediamo che ci possa essere utile poter controllare le singole funzionalità, e ci interessa poter visualizzare altri aspetti generali come per esempio la temperatura esterna ed esterna della casa, oppure visualizzare l'andamento generale dell'impianto fotovoltaico.}
\end{quote}
\clearpage
\noindent\textbf{5. Con quali dispositivi vorresti interagire con il sistema?}\\[1mm]
Ad esempio Computer desktop, oppure dispositivi mobili come telefoni o tablet?\\[1.5mm]
\begin{quote}
\emph{Vorrei poter interagire con il sistema tramite tre telefoni, uno per ciascun componente della famiglia, così ognuno può gestire le funzioni domotiche in modo indipendente. Inoltre, mi piacerebbe accedere da un computer nello studio e dal portatile di nostro figlio, utile per un controllo più dettagliato. Infine, un tablet sarebbe comodo da tenere in una zona comune della casa, come il soggiorno, per un accesso rapido alle funzioni principali.}\\
\end{quote}
\clearpage
\noindent\textbf{6. Quali caratteristiche ritiene importanti per l’interfaccia grafica del sistema?}\\[1.5mm]
\begin{quote}
\emph{Per l’interfaccia grafica, vorrei che fosse chiara e intuitiva, in modo da poter controllare tutto facilmente senza dover perdere tempo a cercare le funzioni. Mi piacerebbe un design moderno, con un’organizzazione ben suddivisa: ad esempio, avere le varie funzionalità separate per categoria (illuminazione, sicurezza, temperatura, ecc.) e una suddivisione per zone della casa, così da poter gestire ogni stanza in modo rapido.}
\end{quote}
\end{document}