\documentclass[italian, 12pt, a4paper]{article}
\usepackage{graphicx} % Required for inserting images
\usepackage[table]{xcolor}

\title{HomeProject}
\author{Matteo Saramin, Filippo Sperandio \textit{ASRL Corporation} \\ {\small ITTS Vito Volterra} \\ \\ \emph{Cliente: Roberto Rossi}}

\newcommand{\version}{Version 1.0}                   %Definisci il comando per la versione
\date{\version\\ 15 Marzo 2025}

\begin{document}

\maketitle                                           %mostra titolo, autore e data
\section{Storico del Documento}
La seguente tabella riporta lo storico delle versioni del documento, includendo la versione, la data di rilascio, l'autore delle modifiche e il nome del file corrispondente.
\begin{center}
    \renewcommand{\arraystretch}{1.5} % Aumenta lo spazio verticale
    \rowcolors{1}{blue!15} % Alterna colori a partire dalla seconda riga
    \begin{tabular}{|c|c|c|c|}
        \hline
        \rowcolor{gray !30}
        \hspace{5mm} Versione \hspace{5mm} & 
        \hspace{5mm} Data \hspace{5mm} & 
        \hspace{5mm} Autore \hspace{5mm} & 
        \hspace{5mm} Nome Documento \hspace{5mm} \\
        \hline
        1.0 & 14/03/'25 & Matteo & Main.tex \\
        \hline
    \end{tabular}
\end{center}    

\section{Introduction}
Questo è il testo della sezione introduttiva.

\end{document}