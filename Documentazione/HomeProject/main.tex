\documentclass[italian, 12pt, a4paper]{article}
\usepackage{graphicx} % Required for inserting images
\usepackage[table]{xcolor}

\title{HomeProject}
\author{Matteo Saramin, Filippo Sperandio \textit{ASRL Corporation} \\ {\small ITTS Vito Volterra} \\ \\ \emph{Cliente: Roberto Rossi}}

\newcommand{\version}{Version 1.0} % Definisce il comando per la versione
\date{\version\\ 15 Marzo 2025}

\begin{document}

\maketitle % Mostra titolo, autore e data

\section{Storico del Documento}
La seguente tabella riporta lo storico delle versioni del documento, includendo la versione, la data di rilascio, l'autore delle modifiche e il nome del file corrispondente.

\begin{center}
    \renewcommand{\arraystretch}{1.5} % Aumenta lo spazio verticale
    \rowcolors{2}{violet!30}{} % Alterna colori a partire dalla seconda riga
    \begin{tabular}{|c|c|c|c|}
        \hline
        \rowcolor{violet!30}
        Versione & Data & Autore & Nome Documento \\
        \hline
        1.0 & 14/03/2025 & Matteo & main.tex \\
        \hline
        2.0&25/03/2025&Matteo e Filippo & main.tex \\
        \hline
    \end{tabular}\\[4mm]
\end{center}
\vspace{15mm}
\section{Sommario}
\begin{enumerate}
    \item Introduzione
    \item Specifiche Requisiti Funzionali
    \item Terzo elemento
\end{enumerate}
\clearpage
\section{Introduzione}
Questo è il testo della sezione introduttiva.
\vspace{20mm}
\section{Specifica Requisiti Funzionali}

\end{document}
